\documentclass[a4paper, 12pt]{article}

\usepackage{upquote}

\begin{document}

\title{Android Notes}
\author{Nazmus Saquib}

\maketitle
\tableofcontents

%%%%%%%%%%%%%%%%%%%%%%%%%%%%%%%%%%%%%%%%%%%%%%%%%
\section{Project Files}
\begin{description}
	\item[src] this folder contains all of our java files
	\item[gen] this folder cotains generated files, we should not need to modify these
	\item[res] folder deals with images, resoources etc. the drawable folders can contain images of different resolution; SDK will use the most appropriate one
	\item[res/layout] contains layout specification of app
	\item[res/values] a number of values folders can be seen, these can be later referenced
	\item[AndroidManifest.xml] maintains everything, boss mode
	\item[res/layout-land] custom folder, layout in case our phone is tilted
	\item[res/raw] custom folder, contains audio
\end{description}

%%%%%%%%%%%%%%%%%%%%%%%%%%%%%%%%%%%%%%%%%%%%%%%%%
\section{Source File Syntax}
\begin{itemize}
	\item \verb|setContentView(R.layout.activity_main)| : sets the layout according to res/layout/activity\_main.xml
\end{itemize}

%-----------------------------------------------%
\subsection{Connection between XML and Java}
\emph{bAdd} is the id as defined in the XML file.
\begin{verbatim}
Button bAdd;
bAdd = (Button) findViewById(R.id.bAdd);
\end{verbatim}

%-----------------------------------------------%
\subsection{Button Click}
Note that argument of \emph{setText} can not be only number.
\begin{verbatim}
add.setOnClickListener(new View.onClickListener(){
	public void onClick(View v){
		counter++;
		dsiplay.setText("Your Total is: " + counter);
	}
});
\end{verbatim}

%-----------------------------------------------%
\subsection{Action Listener Example}
\begin{verbatim}
public void sendMessage(View view){

}
\end{verbatim}
Note that in order to match with the \emph{android:onClick} attribute, the signature must have the following properties:
\begin{itemize}
	\item public method
	\item return value of void
	\item have only one parameter, namely of \emph{View} type
\end{itemize}

%%%%%%%%%%%%%%%%%%%%%%%%%%%%%%%%%%%%%%%%%%%%%%%%%
\section{Layout File for XML}
\begin{itemize}
	\item if a tag does not have another tags inside of it, we can simply use short form of tag close
	\item \verb|android:layout_width="fill_parent"|	: fills the size of the parent
	\item \verb|android:layout_width="wrap_content"| : uses just what is needed
	\item \verb|android:text="@string/ref"| : res/values/strings.xml contains the string ref, defined as say \verb|<string name="ref">this is a reference</string>|, so ref has the value within tags
	\item \verb|android:text="string"| : puts the string within quotes directly
	\item \verb|android:textSize="40sp"| : text size, dp or px should not be used, basically we should use sp for fonts and dp for everything else
	\item \verb|android:layout_gravity="center"| put component at the center, notice we'll need wrap\_content. Otherwise we can also specify \emph{gravity} to be center
	\item \verb|android:@+id/tvDisplay| we are setting up and \emph{id}, and that is \emph{tvDisplay}
\end{itemize}

%%%%%%%%%%%%%%%%%%%%%%%%%%%%%%%%%%%%%%%%%%%%%%%%%
\section{UI Basics}
\begin{itemize}
	\item \emph{ViewGroup}s are container of views such as \emph{LinearLayout}s, \emph{RelativeLayout}s.
	\item \emph{View}s are objects such as \emph{Button}s, \emph{TextView}s etc.
\end{itemize}

%-----------------------------------------------%
\subsection{ViewGroup}
\begin{description}
	\item[LinearLayout] components appear in the order they are written in the XML file.
\end{description}

%-----------------------------------------------%
\subsection{Controls}
\begin{description}
	\item[EditText] user editable text.	
\end{description}

%-----------------------------------------------%
\subsection{Control Attributes}
\begin{description}
	\item[android:id] unique identifier for the view. The \emph{@} sign is required whenever we are referring to any resource object. Plus signs are necessary when creating a resource ID for the first time.
	\item[android:hint] default text to show when the control is blank.
	\item[android:layout\_weight] indicates how much of a free space each component should consume. Once we give a weight, it is better to set \emph{layout\_width} to 0dp, as putting anything else would be irrelevant because the width has to be calculated to match the weight.
	\item[android:onClick] the value is the name of a function in the activity that gets called when the control is clicked.
\end{description}

%%%%%%%%%%%%%%%%%%%%%%%%%%%%%%%%%%%%%%%%%%%%%%%%%
\section{Activity}
Can be simply thought of as a screen. All subclasses of activity must implement \emph{onCreate} method. All activities must be declared in our manifest file using \verb|<activity>| tag. Every activity is invoked by an intent, we can get the intent by invoking \emph{getIntent} method.

%%%%%%%%%%%%%%%%%%%%%%%%%%%%%%%%%%%%%%%%%%%%%%%%%
\section{Intent}
An \emph{Intent} is an object that provides runtime binding between seperate components, such as two activities. We need to import the intent class - \verb|import android.component.Intent|. An Intent can carry a collection of various data types as key-value pairs called \emph{extras}. The \emph{putExtra} method takes the key as first parameter and the value as the second parameter. An activity is started by invoking the \emph{startActivity} method with an Intent as parameter.

%%%%%%%%%%%%%%%%%%%%%%%%%%%%%%%%%%%%%%%%%%%%%%%%%
\section{Command Line}
%-----------------------------------------------%
\subsection{Genymotion}
Make sure to use the path to genymotion, better yet, add to profile / bashrc.
\begin{itemize}
	\item \verb|VBoxManage list vms| shows device name and id
	\item \verb|player --vm-name name-of-device| launches virtual device
	\item \verb|adb install path-to-apk-in-bin| installs apk
	\item \verb|adb uninstall packagename| uninstalls apk
\end{itemize}
%-----------------------------------------------%
\subsection{SDK}
All the tools can be found in \emph{sdk/tools}, the adb has been moved.
\begin{itemize}
	\item \verb|./android avd| launches avd manager, note that once we are using genymotion we won't need this
	\item \verb|./emulator -avd emulator-name| launches the emulator
	\item \verb|adb -s emulator-5554 install apk-path-name| installs the apk in avd, \emph{5554} is the emulator number
	\item \verb|./android list targets| lists virtual devices
	\item The following creates a project with supplied args:
		\begin{verbatim}
			./android create project \
			--target <target_ID> \
			--name <your_project_name> \
			--path path/to/your/project \
			--activity <your_activity_name> \
			--package <your_package_namespace>
		\end{verbatim}
	\item \verb|ant debug| from the project root folder builds the project with debugging
\end{itemize}

\end{document}
