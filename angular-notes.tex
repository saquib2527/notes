\documentclass[a4paper, 12pt]{article}

\usepackage{upquote}

\begin{document}

\title{Angularjs Notes}
\author{Nazmus Saquib}

\maketitle
\tableofcontents

%%%%%%%%%%%%%%%%%%%%%%%%%%%%%%%%%%%%%%%%%%%%%%%%%
\section{Terminologies}
\begin{itemize}
	\item \textbf{directive} - a marker on a html tag that tells angular to run or reference some javascript code
	\item \textbf{modules} - where we write our pieces of Angular app. We also write our dependencies here
	\item \textbf{expressions} - allow us to use dynamic values
	\item \textbf{controllers} - places where we define our app's behavior by defining functions and values
	\item \textbf{filters} - used for formatting values
	\item \textbf{services} - used to retrieve data from etc.
\end{itemize}

%%%%%%%%%%%%%%%%%%%%%%%%%%%%%%%%%%%%%%%%%%%%%%%%%
\section{Modules}
A sample angular module:
\begin{verbatim}
/* app.js */
var app = angular.module('store', []);
\end{verbatim}
Here angular refers to the included library, store is the name of the module and the array is the list of dependencies, in this case empty.

In fact, it is better to wrap our js code in a closure:
\begin{verbatim}
(function(){
	var app = angular.module('store', []);
})();
\end{verbatim}

%%%%%%%%%%%%%%%%%%%%%%%%%%%%%%%%%%%%%%%%%%%%%%%%%
\section{Expression}
Expressions are surrounded by double curly braces.
\begin{verbatim}
I have {{ 4 + 6 }} pens.
\end{verbatim}

%%%%%%%%%%%%%%%%%%%%%%%%%%%%%%%%%%%%%%%%%%%%%%%%%
\section{Directives}
A few directives:
\begin{itemize}
	\item \textbf{ng-app} - defines the module, must be present at the html tag for the page to be an Angular app
	\item \textbf{ng-controller} - defines scope of the controller on a DOM element
	\item \textbf{ng-show} - hides or shows elements
	\item \textbf{ng-hide} - same as show, works on complementary logic
	\item \textbf{ng-repeat} - used for sort of a ``foreach loop''
	\item \textbf{ng-src} - if we are getting the value of src from angular, we need to use this instead of html src property
		\begin{verbatim}
			<img ng-src="product.images[0]">
		\end{verbatim}
	\item \textbf{ng-init} - initializes a value
	\item \textbf{ng-click} - action when clicked
	\item \textbf{ng-class} - excepts an object which has as prperty the name of a class and as value an expression.
		So when the expression is true, the class will be set.
	\item \textbf{ng-model} - binds form element value to the property
	\item \textbf{ng-submit} - allows us to all a function when the form is submitted
	\item \textbf{ng-include} - used for including html snippets, it is worth noting we need to use double quotes and single quotes both -
		in that order
\end{itemize}
%-----------------------------------------------%
\subsection{Custom Directives}
In js file:
\begin{verbatim}
app.directive("productTitle", function(){
	return {
		restrict: 'E',
		templateUrl: 'product-title.html'
	};
});
\end{verbatim}
Here E stands for html element, generally restrict is used to define type of directive.
Other type can be attribute. We use element directive for UI Widgets, whereas attribute directive for mixins, such as tooltips.
templateUrl is used to give the source of the file to be loaded as template.

In html file:
\begin{verbatim}
<product-title><product-title>
\end{verbatim}

We can also use controllers inside our custom directive:
\begin{verbatim}
app.directive("productTitle", function(){
	return {
		restrict: 'E',
		templateUrl: 'product-title.html',
		controller: function(){
			...
		},
		controllerAs: 'alias'
	};
});
\end{verbatim}

%%%%%%%%%%%%%%%%%%%%%%%%%%%%%%%%%%%%%%%%%%%%%%%%%
\section{Controllers}
We need to capitalize the controller name. Assuming the variable app has reference to our module, this sets up a controller:
\begin{verbatim}
var gem = {
	name: "Dodecahedron",
	price: 2.95,
	description: "awesome"
};
app.controller('StoreController', function(){
	this.product = gem;
});
\end{verbatim}
For passing data through controller, we need to set data to be a property of our controller.

We need to specify the html tag upon which the controller should act through an ng-controller directive.
This effectively sets the ``scope'' of the controller. We can use aliases if we want:
\begin{verbatim}
<div ng-controller="StoreController as store">
	<h1>{{ store.product.name }}</h1>
</div>
\end{verbatim}

%%%%%%%%%%%%%%%%%%%%%%%%%%%%%%%%%%%%%%%%%%%%%%%%%
\section{Filters}
Filters are used for formatting. A few filters:
\begin{itemize}
	\item \textbf{currency} - formats numerical values to dollars
	\item \textbf{date} - formats date with the supplied options
	\item \textbf{uppercase} - converts string to all caps
	\item \textbf{limitTo} - limits characters, list items etc. to a specific number
	\item \textbf{orderBy} - field preceded by minus sign means descending order, without sign means ascending 
\end{itemize}

%%%%%%%%%%%%%%%%%%%%%%%%%%%%%%%%%%%%%%%%%%%%%%%%%
\section{Form Validation}
First we need to turn off default html validation through ``novalidate'' property.
\begin{verbatim}
{{ reviewForm.$valid }}
\end{verbatim}
Here ``reviewForm'' is the name of the form.

%%%%%%%%%%%%%%%%%%%%%%%%%%%%%%%%%%%%%%%%%%%%%%%%%
\section{Services}
A few services:
\begin{itemize}
	\item \$http
	\item \$log
	\item \$filter
\end{itemize}
We can attach services to controllers:
\begin{verbatim}
app.controller('SomeController', ['$http', '$log', function($http, $log){
	
}]);
\end{verbatim}


\end{document}
