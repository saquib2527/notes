\documentclass[12pt, a4paper]{article}

\usepackage{upquote}

\begin{document}

\title{Nodejs Notes}
\author{Nazmus Saquib}

\maketitle

%%%%%%%%%%%%%%%%%%%%%%%%%%%%%%%%%%%%%%%%%%%%%%%%%%
\section{The Node Shell (REPL)}
\begin{itemize}
\item to start shell use \textbf{node}
\item to quit shell use \textbf{process.exit(0)}
\end{itemize}

%%%%%%%%%%%%%%%%%%%%%%%%%%%%%%%%%%%%%%%%%%%%%%%%%%
\section{Difference between Browser and Node`s Engine}
\begin{itemize}
\item \textbf{window}, \textbf{localtion}, \textbf{document} are only present in browser
\item \textbf{global} may be present in browsers as window object, node has a lot of functions in global object that are not in chrome
\item \textbf{module} and \textbf{process} are only present in node
\item \textbf{console} object is present in both
\end{itemize}


%%%%%%%%%%%%%%%%%%%%%%%%%%%%%%%%%%%%%%%%%%%%%%%%%%
\section{Requiring Files}
We have to specify the relative path of the file we want to include.
We use the \textbf{require} function to do that, passing as parameter the path.

For core modules or modules installed locally for a project, we don't need to add path - just the name suffices.

We can leave out the ``js'' part of the filename.
In the file being included, we have to assign the ``module.exports'' to whatever we want to include.

Instead of module.exports, we can use \textbf{exports.property1}, \textbf{exports.property2} and so on.

In fact, we can also add a folder, provided it has an ``index.js'' file in it.

To better delineate scope, it is better to always use ``var'' while assigning / declaring variables.


%%%%%%%%%%%%%%%%%%%%%%%%%%%%%%%%%%%%%%%%%%%%%%%%%%
\section{A Minimalistic Node Server}
\begin{verbatim}
var http = require('http');

http.createServer(function(req, res){
	res.writeHead(200, { 'Content-Type': 'text/plain' });
	res.end('Welcome to Node Essentials\n');
}).listen(3000, '127.0.0.1');

console.log('server running on 127.0.0.1:3000');
\end{verbatim}


%%%%%%%%%%%%%%%%%%%%%%%%%%%%%%%%%%%%%%%%%%%%%%%%%%
\section{Event Loop}
The event loop starts when the node script starts.
All I/O in node are non blocking.


%%%%%%%%%%%%%%%%%%%%%%%%%%%%%%%%%%%%%%%%%%%%%%%%%%
\section{NPM}
%------------------------------------------------%
\subsection{Creating a Project}
We can use \textbf{npm init} command to initiate a node project.
\begin{itemize}
\item \textbf{name} - defaults to the folder where the package.json file will reside in.
\item \textbf{version} - version of the project in ``0.0.0'' format.
\item \textbf{description} - description of the project
\item \textbf{entry point} - defaults to index.js
\item \textbf{test command} - we can leave blank if we don't want to perform unit tests
\item \textbf{git repository} - required if we are tryng to distribute our project
\item \textbf{keywords} - again for distributing projects elsewhere
\item \textbf{author} - use \verb|name <email>| format
\end{itemize}

To add modules to our project we use:
\begin{verbatim}
npm install --save express
\end{verbatim}
Invoking install with save records the version of the module being installed in package.json file.
The "dependencies" property of the root level object is an object holding packages and their version.
%------------------------------------------------%
\subsection{Versioning with NPM}
\begin{itemize}
\item \textbf{*} downloads the latest version
\item \textbf{nmp update} is used to get the required version of a package, provided it changed in the mean time.
\end{itemize}
%------------------------------------------------%
\subsection{Miscellaneous use of NPM}
\begin{itemize}
\item We can start a project by invoking \textbf{npm start} command.
\item To uninstall packages first we can delete them from package.json, and issue \textbf{prune} command.
\end{itemize}

\end{document}
