\documentclass[a4paper, 12pt]{article}

\usepackage{upquote}

\begin{document}

\title{Scraping Notes}
\author{Nazmus Saquib}

\maketitle
\tableofcontents

%%%%%%%%%%%%%%%%%%%%%%%%%%%%%%%%%%%%%%%%%%%%%%%%%%
\section{BeautifulSoup}
%------------------------------------------------%
\subsection{installation}
Any of the following commands will install BeautifulSoup4.
\begin{itemize}
\item \verb|sudo apt-get install python-bs4|
\item \verb|sudo pip install beautifulsoup4|
\item \verb|sudo easy_install beautifulsoup4|
\end{itemize}
%------------------------------------------------%
\subsection{basics}
\begin{itemize}
\item first we need to import BeautifulSoup.
\begin{verbatim}
from bs4 import BeautifulSoup
\end{verbatim}
\item BeautifulSoup does not retrieve content from the net. We have to do it ourselves using \emph{urllib2} module.
\begin{verbatim}
import urllib2
content = urllib2.urlopen("http://www.pythonforbeginners.com").read()
\end{verbatim}
\item Now we can create a soup object from the content.
\begin{verbatim}
soup = BeautifulSoup(content)
\end{verbatim}
\item If we are reading a local file we can just use I/O functions.
\begin{verbatim}
soup = open("test.html", 'r').read()
\end{verbatim}
\item To get nicely formatted html use \verb|print(soup.prettify())|
\end{itemize}
%-------------------------------------------------%
\subsection{difference between soup.title and soup.title.string}
\emph{soup.title} returns the title enclosed in title tags, whereas \emph{soup.title.string} returns the string within.
%-------------------------------------------------%
\subsection{retrieving all anchor tags}
\begin{verbatim}
anchors = soup.find_all('a')
for anchor in anchors:
    if anchor.has_attr("href"):
        print(anchor["href"])
\end{verbatim}
Instead of \verb|anchor["href"]|, \verb|anchor.get("href")| would also have worked.
%-------------------------------------------------%
\subsection{miscellaneous}
\begin{description}
\item[soup.tagname] returns the contents of the first matched tag enclosed within the tag
\item[soup.tagname.parent] returns everything within the parent of the tag
\item[soup.tagname.parent.name] returns parent's name
\item[soup.find(id="link3")] finds the element with id of link3
\item[soup.get\_text()] retrieves all the text
\end{description}

%%%%%%%%%%%%%%%%%%%%%%%%%%%%%%%%%%%%%%%%%%%%%%%%%%%
\section{Mechanize}

%%%%%%%%%%%%%%%%%%%%%%%%%%%%%%%%%%%%%%%%%%%%%%%%%%%
\section{Scrapy}

%%%%%%%%%%%%%%%%%%%%%%%%%%%%%%%%%%%%%%%%%%%%%%%%%%%
\section{SQLAlchemy}
%-----------------------------------------------%
\subsection{Installation}
\verb|sudo pip install SQLAlchemy|
%-----------------------------------------------%
\subsection{Version Check}
\begin{verbatim}
>>> import sqlalchemy
>>> sqlalchemy.__version__
'0.9.4'
\end{verbatim}
%-----------------------------------------------%
\subsection{Connecting to a Database (SQLite)}
\begin{verbatim}
>>> from sqlalchemy import create_engine
>>> engine = create_engine("sqlite:///:memory:", echo=True)
\end{verbatim}
\emph{echo=True} for logging.

%%%%%%%%%%%%%%%%%%%%%%%%%%%%%%%%%%%%%%%%%%%%%%%%%
\section{SQLite}
%-----------------------------------------------%
\subsection{Installation}
sudo apt-get install sqlite3 libsqlite3-dev

\end{document}
