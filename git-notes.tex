\documentclass[a4paper, 12pt]{article}

\usepackage{upquote}

\begin{document}

\title{Notes on Git}
\author{Nazmus Saquib}

\maketitle
\tableofcontents

%%%%%%%%%%%%%%%%%%%%%%%%%%%%%%%%%%%%%%%%%%%%%%%%%%
\section{Creating a Repo and Adding to Github}
Make sure ssh keys are set -- https://help.github.com/articles/generating-ssh-keys
\begin{verbatim}
git init
git add filename
git commit -m "the message"
curl -u 'saquib2527' https://api.github.com/user/repos -d '{"name": "reponame"}'
git remote add orgin git@github.com:saquib2527/reponame.git
git push origin master
\end{verbatim}

%%%%%%%%%%%%%%%%%%%%%%%%%%%%%%%%%%%%%%%%%%%%%%%%%%%
\section{Removing a File from Git}
We might nee to force removal with \emph{-f} flag after rm. Last line will be required if we are using a remote repo. Notice this actually removes the file from both repo and file system. To only remove from repo we need to use \emph{--cached} flag after rm.
\begin{verbatim}
git rm filename
git commit -m "removed filename"
git push add origin
\end{verbatim}

\end{document}
