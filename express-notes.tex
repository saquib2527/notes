\documentclass[a4paper, 12pt]{article}

\usepackage{upquote}

\begin{document}

\title{Express Notes}
\author{Nazmus Saquib}

\maketitle
\tableofcontents

%%%%%%%%%%%%%%%%%%%%%%%%%%%%%%%%%%%%%%%%%%%%%%%%%
\section{Installation}
For global installation we can use:
\begin{verbatim}
sudo npm -g install express-generator
\end{verbatim}
Note that we have to use ``sudo''. With the release of express 4 we need to install ``express-genrator'', rather than only ``express''.

%%%%%%%%%%%%%%%%%%%%%%%%%%%%%%%%%%%%%%%%%%%%%%%%%
\section{Getting Help}
\begin{verbatim}
express --help
\end{verbatim}

%%%%%%%%%%%%%%%%%%%%%%%%%%%%%%%%%%%%%%%%%%%%%%%%%
\section{Generating Skeleton App}
\begin{verbatim}
express appName
express appName -c less
\end{verbatim}
The first command simply creates a directory named ``appName'' in the current directory. This directory holds the skeleton framework. The second command does the same, but it also tells to include the less css engine.

We can then install the dependencies and run our app:
\begin{verbatim}
cd appName && npm install
DEBUG=appName ./bin/www
\end{verbatim}

%%%%%%%%%%%%%%%%%%%%%%%%%%%%%%%%%%%%%%%%%%%%%%%%%
\section{A Minimal Skeleton}
We can simply delete the \textbf{public}, \textbf{views} and \textbf{routes} folders. Modified \textbf{app.js}:
\begin{verbatim}
var express = require('express');
var http = require('http');

var app = express();

app.set('port', process.env.PORT || 3000);	

http.createServer(app).listen(app.get('port'), function(){
	console.log("Express Server Listening on Port " + app.get('port'));
});
\end{verbatim}


\end{document}
