\documentclass[a4paper, 12pt]{article}

\usepackage{upquote}

\begin{document}

\title{Laravel Notes}
\author{Nazmus Saquib}

\maketitle
\tableofcontents

%%%%%%%%%%%%%%%%%%%%%%%%%%%%%%%%%%%%%%%%%%%%%%%%%
\section{Setting up Laravel}
%-----------------------------------------------%
\subsection{Installing Laravel}
\begin{verbatim}
cd /var/www
wget https://github.com/laravel/laravel/archive/master.zip
unzip master.zip
mv laravel-master sitename
cd sitename
sudo composer install
sudo chmod -R 777 .
\end{verbatim}
Optionally we could remove master.zip.
Alternatively:
\begin{verbatim}
composer create-project laravel/laravel --prefer-dist /var/www/laravel
\end{verbatim}
%-----------------------------------------------%
\subsection{Granting Permissions}
If the following does not work just use \verb|sudo chmod -R 777 /var/www|
\begin{verbatim}
sudo chown -R $USER:$USER /var/www/laravel
sudo chmod -R 755 /var/www
\end{verbatim}
%-----------------------------------------------%
\subsection{Creating Virtual Hosts}
First we need create a file in \emph{/etc/apache2/sites-avalilable}, for example let us assume the name is \emph{laravel.conf}.
We can open up the file by:
\begin{verbatim}
sudo vim /etc/apache2/sites-available/laravel.conf
\end{verbatim}

Then inside we have to put:
\begin{verbatim}
<VirtualHost *:80>
	ServerName laravel.dev
	ServerAlias www.laravel.dev
	DocumentRoot /var/www/laravel/public
	<Directory /var/www/laravel/public>
		AllowOverride all
	</Directory>
</VirtualHost>
\end{verbatim}
%-----------------------------------------------%
\subsection{Modifying Hosts File}
Open up \emph{/etc/hosts} by invoking \verb|sudo vim /etc/hosts|.
Add the following:
\begin{verbatim}
127.0.0.1    laravel.dev
\end{verbatim}
%-----------------------------------------------%
\subsection{Enable Site and Restart Apache}
Now we have enable the site and restart apache.
\begin{verbatim}
sudo a2ensite laravel.conf
sudo service apache2 restart
\end{verbatim}

\end{document}
