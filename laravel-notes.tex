\documentclass[a4paper, 10pt]{article}

\usepackage{upquote}

\begin{document}

\title{Laravel Notes}
\author{Nazmus Saquib}

\maketitle
\tableofcontents

%%%%%%%%%%%%%%%%%%%%%%%%%%%%%%%%%%%%%%%%%%%%%%%%%
\section{Note to the Reader}
\begin{itemize}
    \item `\textbackslash' inside code denotes continuation of line
\end{itemize}

%%%%%%%%%%%%%%%%%%%%%%%%%%%%%%%%%%%%%%%%%%%%%%%%%
\section{Setting up Laravel}
%-----------------------------------------------%
\subsection{Installing Laravel}
\begin{verbatim}
cd /var/www
wget https://github.com/laravel/laravel/archive/master.zip
unzip master.zip
mv laravel-master sitename
cd sitename
sudo composer install
sudo chmod -R 777 .
\end{verbatim}
Optionally we could remove master.zip.
Alternatively:
\begin{verbatim}
composer create-project laravel/laravel \
    --prefer-dist /var/www/laravel
\end{verbatim}
%-----------------------------------------------%
\subsection{Granting Permissions}
If the following does not work just use

\verb|sudo chmod -R 777 /var/www|
\begin{verbatim}
sudo chown -R $USER:$USER /var/www/laravel
sudo chmod -R 755 /var/www
\end{verbatim}
%-----------------------------------------------%
\subsection{Creating Virtual Hosts}
First we need create a file in \emph{/etc/apache2/sites-avalilable},
for example let us assume the name is \emph{laravel.conf}.
We can open up the file by:
\begin{verbatim}
sudo vim /etc/apache2/sites-available/laravel.conf
\end{verbatim}

Then inside we have to put:
\begin{verbatim}
<VirtualHost *:80>
    ServerName laravel.dev
    ServerAlias www.laravel.dev
    DocumentRoot /var/www/laravel/public
    <Directory /var/www/laravel/public>
        AllowOverride all
    </Directory>
</VirtualHost>
\end{verbatim}
%-----------------------------------------------%
\subsection{Modifying Hosts File}
Open up \emph{/etc/hosts} by invoking \verb|sudo vim /etc/hosts|.
Add the following:
\begin{verbatim}
127.0.0.1    laravel.dev
\end{verbatim}
%-----------------------------------------------%
\subsection{Enable Site and Restart Apache}
Now we have enable the site and restart apache.
\begin{verbatim}
sudo a2ensite laravel.conf
sudo service apache2 restart
\end{verbatim}

%%%%%%%%%%%%%%%%%%%%%%%%%%%%%%%%%%%%%%%%%%%%%%%%%%
\section{Enable Debugging}
In \textbf{app/config/app.php} set debug to true.

%%%%%%%%%%%%%%%%%%%%%%%%%%%%%%%%%%%%%%%%%%%%%%%%%%
\section{Helper Functions}
\begin{itemize}
    \item create a folder named \textbf{libraries} inside app folder.
    \item in the new folder create a file named \textbf{Helper.php}.
    \item put all your static helper functions in this file.
    \item in \textbf{start/global.php} edit the \textbf{ClassLoader::addDirectories} array
        \begin{verbatim}
        add_path().'/libraries'
        \end{verbatim}
    \item use the functions anywhere using \textbf{Helper::function} syntax.
\end{itemize}

%%%%%%%%%%%%%%%%%%%%%%%%%%%%%%%%%%%%%%%%%%%%%%%%%%
\section{Linking Assets}
Assuming we have a css,  a js and an img folder inside of public folder.
%------------------------------------------------%
\subsection{Stylesheets}
\begin{verbatim}
{{ HTML::style('css/custom.css') }}
\end{verbatim}
%------------------------------------------------%
\subsection{Javascripts}
\begin{verbatim}
{{ HTML::script('js/custom.js') }}
\end{verbatim}
%------------------------------------------------%
\subsection{Images}
The parameters are url, alt text and attribute array respectively.
\begin{verbatim}
{{ HTML::image('img/logo.png', 'this is the logo', \
    ['id' => 'logo', 'class' => 'logo']) }}
\end{verbatim}

%%%%%%%%%%%%%%%%%%%%%%%%%%%%%%%%%%%%%%%%%%%%%%%%%%
\section{Generating Links}
\begin{verbatim}
{{ HTML::link('url', 'text', ['class' => 'navbar-brand']) }}
\end{verbatim}

%%%%%%%%%%%%%%%%%%%%%%%%%%%%%%%%%%%%%%%%%%%%%%%%%%
\section{Generating Form Elements}
%------------------------------------------------%
\subsection{Opening and Closing Form}
Opening without parameters sends the form to the same page.
It also assumes post method. We can specify url and method.
\begin{verbatim}
{{ Form::open(['url' => 'process', 'method' => 'put']) }}
{{ Form::close() }}
\end{verbatim}
%------------------------------------------------%
\subsection{Text / Email Input}
Takes three parameters: name, default value and attribute array.
If default value is not desired pass null.
\begin{verbatim}
{{ Form::text('name', 'default value', \ 
    ['class' => 'form-control']) }}
{{ Form::email('email', null, ['placeholder' => 'email']) }}
\end{verbatim}
%------------------------------------------------%
\subsection{Password Input}
Takes two parameters: name and attribute array.
\begin{verbatim}
{{ Form::password('password', ['class' => 'form-control']) }}
\end{verbatim}
%------------------------------------------------%
\subsection{Submit Input}
takes two parameters: text on button and attribute array.
\begin{verbatim}
{{ Form::submit('Submit', ['class' => 'btn']) }}
\end{verbatim}

%%%%%%%%%%%%%%%%%%%%%%%%%%%%%%%%%%%%%%%%%%%%%%%%%%
\section{Check if User is Logged in}
\begin{verbatim}
if(Auth::check()){
    echo 'logged in';
}
if(Auth::guest()){
    echo 'no user logged in';
}
\end{verbatim}
%------------------------------------------------%
\subsection{Check User Attributes}
First make sure user is logged in, then use the following:
\begin{verbatim}
if(Auth::user()->type == 'A'){
    echo 'admin';
}
\end{verbatim}
Notice if the check is not made we will get an error saying we are
trying to access property of non-object.

%%%%%%%%%%%%%%%%%%%%%%%%%%%%%%%%%%%%%%%%%%%%%%%%%%
\section{Filters}
%------------------------------------------------%
\subsection{Creating Filters}
One suitable place for declaring filters is \textbf{app/filters.php}.
Here is a filter that checks whether user is admin, provided a user is
in fact logged in.
\begin{verbatim}
Route::filter('admin', function
{
    if(Auth::check()){
        if(Auth::user()->type == 'A') return;
    }
    return Redirect::to('/')->with([
        'flashMessage' => 'access denied',
        'alertClass' => 'alert-danger'
    ]);
});
\end{verbatim}
%------------------------------------------------%
\subsection{Adding Filters before Controller Functions}
\begin{verbatim}
public function __construct()
{
    $this->beforeFilter('auth');
    $this->beforeFilter('admin',  \
        ['only' => ['dashboard', 'manage']]);
}
\end{verbatim}

%%%%%%%%%%%%%%%%%%%%%%%%%%%%%%%%%%%%%%%%%%%%%%%%%%
\section{Form Validation}
%------------------------------------------------%
\subsection{Catch Inputs}
\begin{verbatim}
$inputs = Input::only('username', 'password');
\end{verbatim}
%------------------------------------------------%
\subsection{Define Rules}
Rules can be defined models as static variables to keep the controller slim.
\begin{verbatim}
public static $rules = [
    'email' => 'required|email|unique:users,email',
    'password' => 'required|min:6',
    'password2' => 'required|same:password'
];
\end{verbatim}
%------------------------------------------------%
\subsection{Create Validator Object}
\begin{verbatim}
$v = Validator::make($inputs, User::$rules);
if($v->passes()) echo 'validation passed';
if($v-fails()) echo 'failed';
\end{verbatim}

%%%%%%%%%%%%%%%%%%%%%%%%%%%%%%%%%%%%%%%%%%%%%%%%%%
\section{Login Logout}
%------------------------------------------------%
\subsection{Login}
Use the function \textbf{Auth::attempt()} to login.
If successfull it returns true value.
\begin{verbatim}
if(Auth::attempt($inputs)) return Redirect::intended('/');
\end{verbatim}
%------------------------------------------------%
\subsection{Logout}
\begin{verbatim}
Auth::logout();
return Redirect::to('/');
\end{verbatim}

\end{document}
