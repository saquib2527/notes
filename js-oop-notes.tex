\documentclass[12pt, a4paper]{article}

\usepackage{upquote}

\begin{document}

\title{JS Notes on OOP and Functions}
\author{Nazmus Saquib}

\maketitle
\tableofcontents

%%%%%%%%%%%%%%%%%%%%%%%%%%%%%%%%%%%%%%%%%%%%%%%%%%
\section{Intro to JS Objects}
JS Objects are said to be classless / dynamic, so as to say they are not restricted by their class definition,
 and may change during the duration of execution.

%%%%%%%%%%%%%%%%%%%%%%%%%%%%%%%%%%%%%%%%%%%%%%%%%%
\section{Object Creation}
%------------------------------------------------%
\subsection{Constructor Function}
Define functions on prototype, rest on the constructor. Otherwise, every time we create an object,
the function also gets created.
\begin{verbatim}
function Person(fname, lname){
	this.fname = fname;
	this.lname = lname;
}
Person.prototype.get_name = function(){
	return this.fname + " " + this.lname;
}
\end{verbatim}
%------------------------------------------------%
\subsection{Protect Global Scope}
Objects need to be created using the following syntax:
\begin{verbatim}
var jill = new Person(fname, lname);
\end{verbatim}
If we leave out the new keyword, the object is scoped globally. To protect against this, we can add 
a check at the beginning of the constructor function:
\begin{verbatim}
if(this === window){
	return new Person(fname, lname);
}
\end{verbatim}
%------------------------------------------------%
\subsection{Prototypal Revisited}
\begin{verbatim}
/*protecting against browser that dont support create*/
if(typeof Object.create !== 'function'){
	Object.create = function(o){
		function F(){}
		F.prototype = o;
		return new F();
	};
}

var person = {
	fname: '',
	lname: ''
};
var jill = Object.create(person);
\end{verbatim}
%------------------------------------------------%
\subsection{Creating Generic Object}
\begin{verbatim}
var obj1 = new Object();
\end{verbatim}
Notice the variable is actually the reference to the object, not container for the object as in primitive types.

\begin{verbatim}
var obj2 = obj1;
\end{verbatim}
Here obj2 references the same object as obj1.

\begin{verbatim}
obj1 = null;
\end{verbatim}
For dereferencing objects we can assign them to null.
%------------------------------------------------%
\subsection{Literal Form of Object Creation}
\begin{verbatim}
var book = {
    name: "A Book",
    year: "2014"
};
\end{verbatim}
It is possible to use string literals as property names, which we actually need if the property has spaces or special characters.
%------------------------------------------------%
\subsection{Adding Properties}
JS allows us to add properties to objects whenever we want.
\begin{verbatim}
obj1.prop1 = "property 1";
\end{verbatim}

%%%%%%%%%%%%%%%%%%%%%%%%%%%%%%%%%%%%%%%%%%%%%%%%%%
\section{The this Object}
Every scope in JavaScript has a this object that represents the calling object for the function.
In the global scope, this represents the global object (window in web browsers).
When a function is called while attached to an object, the value of this is equal to that object by default.
The following code snippet illustrates the notion:
\begin{verbatim}
function sayName(){
    console.log(this.name);
}

var person1 = {
    name: "John",
    say: sayName
};

var person2 = {
    name: "Charles",
    say: sayName
};

var name = "Michael";

person1.say(); //John
person2.say(); //Charles
sayName(); //Michael
\end{verbatim}

%%%%%%%%%%%%%%%%%%%%%%%%%%%%%%%%%%%%%%%%%%%%%%%%%%
\section{Properties}
JS has \textbf{own properties} and \textbf{prototype properties}. Own properties are assigned as stated previously.
To check any kind of property of an oject we use:
\begin{verbatim}
if("prop" in obj){}
\end{verbatim}
To check existence of own property we use:
\begin{verbatim}
if(obj.hasOwnProperty("prop")){}
\end{verbatim}
We can also remove a property:
\begin{verbatim}
delete obj.prop;
\end{verbatim}
All properties (both prototye and own) added to an object are enumerable,
i.e. we can iterate over them using a for in loop.
\begin{verbatim}
var prop;
for(prop in obj){
    console.log("Name: " + prop);
    console.log("Value: " + obj[prop] );   
}
\end{verbatim}
It is possible to retrieve an enumerable list of own property names:
\begin{verbatim}
var props = Object.keys(obj);
\end{verbatim}

%%%%%%%%%%%%%%%%%%%%%%%%%%%%%%%%%%%%%%%%%%%%%%%%%%
\section{Inheritance}
%------------------------------------------------%
\subsection{Through prototype}
Assuling Person and SuperHuman are already defined:
\begin{verbatim}
SuperHuman.prototype = new Person();
SuperHuman.prototype.constructor = SuperHuman;
\end{verbatim}
Notice as we are redefining the prototype, we need to make sure constructor is
assigned seperately; as it will get overridden too.

%%%%%%%%%%%%%%%%%%%%%%%%%%%%%%%%%%%%%%%%%%%%%%%%%%
\section{Functions}
Functions are actually objects in JS. The internal property \textbf{Call} differentiates functions form other objects. As functions are first-class citizens in JS, we could:
\begin{itemize}
\item assign them to variables
\item add them to objects
\item pass them to other functions as arguments
\item return them from functions
\item basically use them wherever we can use any other reference value
\end{itemize}
%------------------------------------------------%
\subsection{Function Declaration}
\begin{verbatim}
function add(num1, num2){
    return num1 + num2;
}
\end{verbatim}
Function declarations are hoisted to the top of the context. So basically we could first use a function and then declare it.
%------------------------------------------------%
\subsection{Function Expression}
\begin{verbatim}
var add = function(n1, n2){
    return n1 + n2;
};
\end{verbatim}
%------------------------------------------------%
\subsection{Function Parameters}
We can pass any number of parameters to any function without causing error.
Function parameters are stored in array like structure called \textbf{arguments}.
The \textbf{length} prperty tells how many arguments are present. 
The \textbf{arguments} prperty is automatically present inside a function.
The length property of a function defines its expected arity.
%------------------------------------------------%
\subsection{Function Overloading}
Not really possible in JS, the latest signature overwrites the previous ones.
%------------------------------------------------%
\subsection{Changing this}
The \emph{call} function has the following format:
\begin{verbatim}
functionName.call(context, parameter)
\end{verbatim}
The \emph{apply} has similar signature, but it accepts an array as the second parameter.

\end{document}
