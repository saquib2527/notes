\documentclass[a4paper, 12pt]{article}

\usepackage{upquote}

\begin{document}

\title{Bash Notes}
\author{Namus Saquib}

\maketitle
\tableofcontents

%%%%%%%%%%%%%%%%%%%%%%%%%%%%%%%%%%%%%%%%%%%%%%%%%
\section{Finding a Terminal}
\begin{itemize}
	\item \verb|<Ctrl><Alt><t>| opens up terminal, \verb|<Alt><Space>, x| maximizes the terminal
	\item \verb|<Ctrl><Alt><F#>| opens up full screen terminal, F7/F8 is the GUI
\end{itemize}

%%%%%%%%%%%%%%%%%%%%%%%%%%%%%%%%%%%%%%%%%%%%%%%%%
\section{Changing Directory}
\begin{itemize}
	\item \verb|cd| goes to home
	\item \verb|cd $HOME| goes to home
	\item \verb|cd ~| goes to home
	\item \verb|.| is the current directory
	\item \verb|..| is the parent directory
	\item parent directory of the root is root itself
\end{itemize}

%%%%%%%%%%%%%%%%%%%%%%%%%%%%%%%%%%%%%%%%%%%%%%%%%
\section{Creating and Deleting Directories}
%-----------------------------------------------%
\subsection{Create Directory}
\begin{verbatim}
mkdir dir-name
\end{verbatim}
If we want to create nested directories and we are not sure whether the parent folders are present, we can use \emph{p} option.
\begin{verbatim}
mkdir -p parent1/parent2/dir-name
\end{verbatim}

%%%%%%%%%%%%%%%%%%%%%%%%%%%%%%%%%%%%%%%%%%%%%%%%%
\section{Renaming/Moving Files}
%-----------------------------------------------%
\subsection{Renaming a File}
We can rename file by invoking \verb|mv oldname newname|.
To guard against cases where there already exists a file named \emph{newname}, we can use the \emph{i} option.
\begin{verbatim}
mv -i oldname newname
\end{verbatim}
%-----------------------------------------------%
\subsection{Move Files}
We can move multiple files to a folder. For this we invoke \emph{mv} with arguments the last of which is destinaton folder.
\begin{verbatim}
mv file1 file2 file3 ~/stuff
\end{verbatim}
The above moves three files to \emph{stuff} folder inside home directory.

%%%%%%%%%%%%%%%%%%%%%%%%%%%%%%%%%%%%%%%%%%%%%%%%%
\section{Working with Processes}
\begin{itemize}
	\item \verb|top| shows currently running/idle processes
	\item \verb|pgrep process-name| shows process id of the process
	\item \verb|kill -KILL pid| kills the process by telling the os to deal with it
	\item \verb|kill -9 pid| same as previous
	\item \verb|kill pid| tries to kill by signalling the process
	\item \verb|kill -l| lists all signals possible to send with kill
\end{itemize}

%%%%%%%%%%%%%%%%%%%%%%%%%%%%%%%%%%%%%%%%%%%%%%%%%
\section{Wget}
We can omit the capital o option it we are not interested in renaming the file
\begin{verbatim}
wget -O filename url
\end{verbatim}

%%%%%%%%%%%%%%%%%%%%%%%%%%%%%%%%%%%%%%%%%%%%%%%%%
\section{Miscellaneous}
%-----------------------------------------------%
\subsection{Delete Files with Size of 0}
\verb|find folder-name -size 0 -delete|
%-----------------------------------------------%
\subsection{See Ports in Use}
\verb|sudo netstat -tulpn|
%-----------------------------------------------%
\subsection{Check Distro Version}
\verb|lsb_release -a|
%-----------------------------------------------%
\subsection{Install/Uninstall/Reconfigure .deb Files}
\begin{verbatim}
sudo dpkg -i filename.deb
sudo dpkg -r packagename
sudo dpkg-reconfigure packagename
\end{verbatim}
%-----------------------------------------------%
\subsection{Working with Partitions}
\begin{itemize}
	\item to check mounted partitions use \verb|df -aTh|. Notice the mounted partitions can be accessed through the path \verb|/media/New Volume|, \verb|/media/New Volume_| and so on. So we are basically adding underscores for each partition mounted.
	\item \verb|sudo fdisk -l| shows all the devices.
	\item to mount a device use the following:
		\begin{verbatim}
			mkdir "media/New Volume"
			sudo mount -t ntfs /dev/sda3 "/media/New Volume"
		\end{verbatim}
		The \emph{-t} stands for type of partition, we can use the previous command to find that. We can then access the drive in the following way:
		\begin{verbatim}
			cd "/media/New Volume"
		\end{verbatim}
	\item to unmount use \verb|umount /dev/sda3|.
\end{itemize}

\end{document}
