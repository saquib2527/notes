\documentclass[12pt, a4paper]{article}

\usepackage{upquote}

\begin{document}

\title{JS Functions}
\author{Nazmus Saquib}

\maketitle
\tableofcontents

%%%%%%%%%%%%%%%%%%%%%%%%%%%%%%%%%%%%%%%%%%%%%%%%%%
\section{Immediate Execution}
\begin{verbatim}
var plus = function(a, b){
    return a + b;
}();
\end{verbatim}
The parentheses at the end makes it immediately invokable.
The above code will actually return \textbf{NaN}.
To get a result we can pass parameters within parentheses.

%%%%%%%%%%%%%%%%%%%%%%%%%%%%%%%%%%%%%%%%%%%%%%%%%%
\section{Invoking Functions}
%------------------------------------------------%
\subsection{Traditional Invocation}
\begin{verbatim}
var plus = function(a, b){
    return a + b;   
};

plus();
\end{verbatim}
In this case the \textbf{this} reference is bound to global scope.
%------------------------------------------------%
\subsection{Oddity of return}
Return can actually be considered as a function.
\begin{verbatim}
function plus(a, b){
    return (
        console.log(a + b),
        console.log(this),
        console.log(arguments)
    );
}

plus(2, 2);
\end{verbatim}
A point to note is \textbf{arguments} is an array like object that is available for every functions.
Notice this is not actually an array, as it devoids a lot of its functions.
%------------------------------------------------%
\subsection{Difference in \_\_proto\_\_}
\begin{verbatim}
var Person = function(){
    var name;
    var ssn;    
};

fPerson = new Person();

console.dir(fPerson);
\end{verbatim}
Here we see \textbf{\_\_proto\_\_} shows \textbf{Object}.

\begin{verbatim}
function Person(){
    var name;
    var ssn;    
};

fPerson = new Person();

console.dir(fPerson);
\end{verbatim}
Here we see \textbf{\_\_proto\_\_} shows \textbf{Person}.

%%%%%%%%%%%%%%%%%%%%%%%%%%%%%%%%%%%%%%%%%%%%%%%%%%
\section{Prototype Extension}
JS has prototypal inheritance.
\begin{verbatim}
var speak = function(message){
    console.log(message);
};

var Dog = function(){
    var name;
    var breed;
};

Dog.prototype.speak = speak;

var rover = new Dog();
rover.speak('woof'); //outputs 'woof' in console
\end{verbatim}
The benefit of extending function in this way is we can add the function speak to other classes.

%%%%%%%%%%%%%%%%%%%%%%%%%%%%%%%%%%%%%%%%%%%%%%%%%%
\section{call and apply}
\textbf{call} and \textbf{apply} allow us to change the scope of the function. It follows this syntax:
\begin{verbatim}
//singleArgument can be object or primitive
functionName.call(thisReference, singleArgument); 
functionName.apply(thisReference, arrayOfArguments);
\end{verbatim}

%%%%%%%%%%%%%%%%%%%%%%%%%%%%%%%%%%%%%%%%%%%%%%%%%%
\section{arguments}
\textbf{arguments} is an array like parameter passed to all functions implicitly.
We can random access this data structure, we can also find length through the \textbf{length} property.
But we can not use other array functions such as pop etc.

%%%%%%%%%%%%%%%%%%%%%%%%%%%%%%%%%%%%%%%%%%%%%%%%%%
\section{Anonymous Closure / Self Executing Function}
\begin{verbatim}
(function(){
    var a = "hello";
    console.log("anonymous closure!!");
})();
\end{verbatim}
Notice variable a's scope is constrained within the anonymous closure
i.e. a will not be recognized outisde the function.

%%%%%%%%%%%%%%%%%%%%%%%%%%%%%%%%%%%%%%%%%%%%%%%%%%
\section{Hoisting and Scope}
\begin{itemize}
\item Scope within function. In the following code the scope of dogName is restricted within the function.
\begin{verbatim}
function myDog(){
    var dogName = "Rover";  
}
console.log(dogName); //error
\end{verbatim}
\item any variable created without the var keyword becomes global.
\item nested function looks for a variable inside that function. If none is available, it looks for one in the parent.
\item variable declarations (not assignments) are hoisted.
\begin{verbatim}
function myDog(){
    console.log(dogName + " says woof!");
    var dogName = "Fido";   
}
myDog();
\end{verbatim}
The above would output ``undefined says woof!''. The above is actually translated internally as:
\begin{verbatim}
function myDog(){
    var dogName;
    console.log(dogName + " says woof!");
    dogName = "Fido";   
}
\end{verbatim}
\item function definitions also get hoisted, meaning we can call a function before its definition.
\end{itemize}

%%%%%%%%%%%%%%%%%%%%%%%%%%%%%%%%%%%%%%%%%%%%%%%%%%
\section{Creating and Namespacing Modules}
Basically we assign a self executing function to a variable.
This way our module variables do not conflict with outsider ones.
To communicate with the outside world we have to use return statement.
\begin{verbatim}
var myModule = (function(){
    return {
        speak: function(){
            console.log("hello");
        }
    };  
})();

myModule.speak();
\end{verbatim}
%------------------------------------------------%
\subsection{Passing Values to Functions}
It is better to pass values as an object rather than single entities.
In this way we can pass multiple values.
\begin{verbatim}
var myModule = (function(){
    return {
        speak: function(){
            console.log(arguments[0].say);
        }
    };
})();

myModule.speak({say: "woof!"});
\end{verbatim}

The above code suffers from a potential risk of error.
If the user does not pass any argument, we will get an error.
To fix this inside the speak function we can write:
\begin{verbatim}
var myArgs = arguments[0] || '';
var statement = myArgs.say || 'hello';
console.log(statement);
\end{verbatim}
%------------------------------------------------%
\subsection{Default Values}
Even a better way would be to set up a DEFAULTS object in our module.
\begin{verbatim}
var myModule = (function(){
    var DEFAULTS = {
        say: "hello"
    };

    return {
        speak: function(){
            var myArgs = arguments[0] || {};
            var statement = myArgs.say || DEFAULTS.say; 
            console.log(statement);
        }
    };
})();

myModule.speak(); // hello
myModule.speak({ say: "hi" }); //hi
\end{verbatim}

%%%%%%%%%%%%%%%%%%%%%%%%%%%%%%%%%%%%%%%%%%%%%%%%%%
\section{Chaining Multiple Methods}
\begin{verbatim}
var myModule = (function(){
    var DEFAULTS = {
        say: "hello",
        speed: "normal"
    };

    return {
        speak: function(){
            var myArgs = arguments[0] || {};
            var statement = myArgs.say || DEFAULTS.say; 
            console.log(statement);
            //makes chainable
            return this; //returns myModule object
        },
        run: function(){
            var myArgs = arguments[0] || {};
            var running = myArgs.speed || DEFAULTS.speed;   
            console.log("running..." + running);
            //makes chainable
            return this; //returns myModule object
        }
    };
})();

myModule.speak().run({
    speed: "fast"
}).speak({
    say: "whatever bro"
}).run();
//hello
//running...fast
//whatever bro
//running...normal
\end{verbatim}


\end{document}
