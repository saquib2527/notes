\documentclass[10pt, a4paper]{article}

\usepackage{upquote}

\begin{document}

\title{Django Notes}
\author{Nazmus Saquib}

\maketitle
\tableofcontents

%%%%%%%%%%%%%%%%%%%%%%%%%%%%%%%%%%%%%%%%%%%%%%%%%%
\section{Starters}
%------------------------------------------------%
\subsection{Install Django}
\begin{verbatim}
sudo pip install -U django==1.5.4
\end{verbatim}
%------------------------------------------------%
\subsection{Check Django Version}
\begin{verbatim}
python -c "import django; print(django.get_version())"
\end{verbatim}
%------------------------------------------------%
\subsection{Create Project and Run Server}
\begin{verbatim}
django-admin.py startproject tango_with_django_project
cd tango_with_django_project
python manage.py runserver
\end{verbatim}
Browse to \textbf{127.0.0.1:8000} to view initial web page.
\textbf{Ctrl+C} stops server.


%%%%%%%%%%%%%%%%%%%%%%%%%%%%%%%%%%%%%%%%%%%%%%%%%%
\section{Django Application}
A Django project is a collection of configurations and
applications that together make up a given web application
or website. This results in reusable software components.
\begin{verbatim}
python manage.py startapp rango
\end{verbatim}
To make our django project aware of this application we
need to modify our \textbf{settings.py} file. In the
tuple \textbf{INSTALLED\_APPS} we need to add the name
of the newly created application, in this case `rango'.


%%%%%%%%%%%%%%%%%%%%%%%%%%%%%%%%%%%%%%%%%%%%%%%%%%
\section{Views}
Our application folder has a \textbf{views.py} file
that we can manipulate for setting views.
\begin{verbatim}
from django.http import HttpResponse

def index(request):
    return HttpResponse('Rango says hello world!')
\end{verbatim}
Now we must map a \textbf{URL} to the view.
%------------------------------------------------%
\subsection{Mapping URLs}
\begin{itemize}
\item mappings go into urls.py file
\item project config folder has one, apps can also have theirs
\item we could add all patterns to project config folder
\item best practice is to divide the mappings and assembling
\item this ensures loose coupling
\item a tuple must be present named \textbf{urlpatterns}
\item it contains calls to django.confs.urls.url()
\item first param to url is the pattern
\item second tells which view (function) to render
\item third param is optional, used to distinguish one mapping from another
\item to make our project aware of apps mapping we need to modify urls.py
in project config folder. Basically we need to add the following as the 
last member of \textbf{urlpatterns} tuple.
\begin{verbatim}
url(r'^rango/', include('rango.urls')),
\end{verbatim}
\item in the previous case anything after rango in url is handled by app
specific urls
\end{itemize}
Sample mapping of rango app:
\begin{verbatim}
from django.conf.urls import patterns, url
from rango import views

urlpatterns = patterns('',
        url(r'^$', views.index, name='index'))
\end{verbatim}

\end{document}
